% ---------------------------- Problem 1----------------------------------
\subsubsection*{\center Задача № 1.}
{\bf Условие.~}
Разложить в ряд Фурье заданную функцию $f(x)$, построить графики $f(x)$ и суммы ее ряда Фурье. Если не указывается, какой вид разложения в ряд необходимо представить, то требуетчя разложить функцию либо в общий тригонометрический ряд Фурье, либо следует выбрать оптимальный вид разложения в зависимости от данной функции.


\\  &
	$
	f(x)=\left\{
	\begin{array}{r}
	x, \quad0 \leq x \leq \dfrac{\pi}{2}, \\
	\dfrac{\pi}{2}, \quad\dfrac{\pi}{2}<x \leq \pi,
	\end{array}
	\right.
	$
	на отрезке $[0;\,\pi]$ по косинусам кратных дуг.

{\bf Решение.~}	
%График
\begin{center}
	\begin{tikzpicture}

	\begin{axis}[xmin=-0.5,	xmax=3.5, 	ymin=-0.5,	ymax=2.5,
	xtick = {0,1.57,3.14},
	ytick = {-1.57,0,1.57},
	width=0.5\textwidth,
	height=0.4\textwidth,
	axis x line=middle,
	axis y line=middle, 
	every axis x label/.style={at={(current axis.right of origin)},anchor=west},
	every inner x axis line/.append style={|-latex'},
	every inner y axis line/.append style={|-latex'},
	minor tick num=1,			
	axis equal=true,
	xlabel=$x$, 
	ylabel=$y$,          
	samples=100,
	clip=true,
	]\addplot[color=black, line width=1.5pt,domain=0:1.57] {\x};

	\addplot[color=black, line width=1.5pt,domain=1.57:3.14]{1.57};
	\addplot[thick,dashed] coordinates {(1.57,0) (1.57,1.57)};

	\addplot[
	mark=*,
	mark options={fill=black, draw=black},
	only marks,
	] coordinates {(2, -1)};

	
	\end{axis}
	\end{tikzpicture}
\end{center}
\noindent
Построим тригонометрический ряд Фурье вида
$$
f(x)=\frac{a_0}{2}+\sum_{n=1}^\infty 
	a_n\cos{(n\omega x)},\quad\text{где} \,\,\omega=\frac{2\pi}{T},\,T=\pi.
$$
\noindent
Вычислим коэффициенты
$$
\begin{array}{rcl}
a_0 &=& \displaystyle\frac{2}{\pi}\left(
\int\limits_0^\frac{\pi}{2}
x\,dx + \int\limits_\frac{\pi}{2}^\pi
\frac{\pi}{2}\,dx \right) = 
\frac{2}{\pi}  \biggl( \frac{x^2}{2} \biggr|_0^\frac{\pi}{2}
+  \frac{\pi^2}{4} \biggr)  = \frac{3\pi}{4},												\\[12pt]
a_n &=& \displaystyle\frac{2}{\pi}\left(
	\int\limits_0^\frac{\pi}{2}
	x \cos nx\,dx + \int\limits_\frac{\pi}{2}^\pi
	\frac{\pi}{2}\cos nx\,dx \right) ={}									\\[12pt]
	&=& \displaystyle\frac{2}{\pi}\left(
	\frac{x\sin nx}{n} \right|_0^\frac{\pi}{2}
	- \int_0^\frac{\pi}{2} \frac{\pi}{2} \cos nx \, dx
	+\frac{\pi}{2n} \sin nx \left. \biggr|_\frac{\pi}{2}^\pi
	\right) = 	\\[12pt]
	&=& \displaystyle\frac{2}{\pi n^2}\left(\cos\frac{\pi n}{2}-1 \right),	\\[12pt]
\end{array}
$$
Применив теорему Дирихле о поточечной сходимости ряда Фурье, видим, что построенный ряд Фурье сходится 
к периодическому (с периодом $T=2\pi$) продолжению исходной функции при всех $x$ 
График функции $S(x)$, где $S(x)$ --- сумма ряда Ферье, имеет следующий вид
\begin{center}
	\begin{tikzpicture}
	\begin{axis}[xmin=-6.28, xmax=12.56, ymin=-1, ymax=2,
	xtick = {-6.26,-4.71,-1.57,1.57,4.71,6.28,7.85,10.99,12.56},
	ytick = {0,1.57,3.14},
	width=1\textwidth,
	height=0.4\textwidth,
	axis x line=middle,
	axis y line=middle, 
	every axis x label/.style={at={(current axis.right of origin)},anchor=west},
	every inner x axis line/.append style={|-latex'},
	every inner y axis line/.append style={|-latex'},
	minor tick num=1,			
	axis equal=true,
	xlabel=$x$, 
	ylabel=$S(x)$,          
	samples=100,
	clip=true,
	]
	\addplot[color=black, line width=1.5pt,domain=-6.23:-4.71] {\x + 6.28};
	\addplot[color=black, line width=1.5pt,domain=-4.71:-1.57]{1.57};
	\addplot[color=black, line width=1.5pt,domain=-1.57:0] {-\x};
	\addplot[color=black, line width=1.5pt,domain=0:1.57]{\x};
	\addplot[color=black, line width=1.5pt,domain=1.57:4.71] {1.57};
	\addplot[color=black, line width=1.5pt,domain=4.71:6.28]{6.28-\x};
	\addplot[color=black, line width=1.5pt,domain=6.28:7.85] {\x - 6.28};
	\addplot[color=black, line width=1.5pt,domain=7.85:10.99]{1.57};
	\addplot[color=black, line width=1.5pt,domain=10.99:12.56]{12.56-\x};
	\addplot[thick,dashed] coordinates {(-4.71,0) (-4.71,1.57)};
	\addplot[thick,dashed] coordinates {(-1.57,0) (-1.57,1.57)};
	\addplot[thick,dashed] coordinates {(1.57,0) (1.57,1.57)};
	\addplot[thick,dashed] coordinates {(4.71,0) (4.71,1.57)};
	\addplot[thick,dashed] coordinates {(7.85,0) (7.85,1.57)};
	\addplot[thick,dashed] coordinates {(10.99,0) (10.99,1.57)};
	
	\end{axis}
	\end{tikzpicture}
\end{center}
\noindent
\textbf{Ответ:}
$$
&f(x) = \frac{3\pi}{8} + \sum_{n=1}^\infty \frac{2}{\pi n^2}\left(\cos\frac{\pi n}{2} - 1 \right)  \cos nx $$






% ---------------------------- Problem 2----------------------------------
\subsubsection*{\center Задача № 2.}
{\bf Условие.~}
Для заданной графически функции $y(x)$ построить ряд Фурье в комплексной форме, изобразить график суммы построенного ряда

%График
\begin{center}

	\begin{tikzpicture}[
	declare function={
		func(\x)= (\x < 0) * (0)   			+
		and(\x >= 0, \x <= 1) * (\x)     	+
		and(\x >  1, \x <= 2) * (0.5)		+
		(\x >= 2) * (0) ;
	}
	]
	\begin{axis}[
	axis x line=middle, axis y line=middle,
	axis equal,	
	ymin=-0.5, ymax=1.1, ytick={-0.5,0,0.5,1,1.5}, ylabel=$y$,
	xmin=-1.1, xmax=3.0, xtick={-1,0,1,2}, xlabel=$x$,
	domain=-0.0:2.0,samples=600, % added
	height = 300,
	]
	
	\addplot [domain=0:1,blue,line width=2pt] {\x};
	\addplot [domain=1:2,blue,line width=2pt] {0.5};
	\addplot [dashed, black] coordinates {(1,0)(1,1.0)};
	\addplot [dashed, black] coordinates {(2,0)(2,0.5)};
	\end{axis}
	\end{tikzpicture}

\end{center}

\noindent
\textbf{Решение.}\\

\noindent
Ряд Фурье в комплексной форме имеет следующий вид
\[
f(x) = \sum_{n=-\infty}^\infty c_n e^{i\omega nx},\quad c_n=\frac{1}{T}\int\limits_a^b f(x) e^{-i\omega nx}dx,\,\omega=\frac{2\pi}{T}.
\]
В нашем примере $ a=0,b=2,T=2,\omega=\pi$, 
найдем коэффицинеты $c_n,\,n=0,\pm1,\pm2,\ldots$





\begin{align}
c_0 &=\displaystyle\frac{1}{2} \left(\int\limits_0^1 xdx + 
\int\limits_1^2 \frac{1}{2} dx \right) =\frac{1}{4}+\frac{1}{4}=\frac{1}{2} ,\nonumber \\[12pt]
c_n &=\displaystyle\frac{1}{2}
\left(\int\limits_0^1 x e^{-i\pi nx} dx
+ \int\limits_1^2 \frac{1}{2} e^{-i\pi nx}dx \right) =\nonumber \\[12pt]
&= \left.\frac{x e^{-i\pi nx}}{-2i\pi n}\right|_0^1 + 
\int\limits_0^1 \frac{ e^{-i\pi nx}}{2i\pi n}dx +
\left.\frac{e^{-i\pi nx}}{-4i\pi n}\right|_1^2 = 
-\frac{e^{-i\pi n}+e^{-2i\pi n}}{4i\pi n}+
\frac{e^{-i\pi n}-1}{2\pi^2 n^2} = \nonumber\\
&= \frac{\left(\left(-1\right)^n-1\right)\left(2i-\pi n\right)}{4i\pi^2n^2}.\nonumber
\end{align}

\noindent
Применив теорему Дирихле о поточечной сходимости ряда Фурье, видим, что построенный ряд Фурье сходится 
к периодическому (с периодом $T=3$) продолжению исходной функции при всех $x\ne 3n$, и $S(3n)=-1/2$ при 
$n=0,\pm1,\pm2,\ldots$, где $S(x)$ --- сумма ряда Фурье. График функции $S(x)$ имеет вид
\begin{center}
	\begin{tikzpicture}
	\begin{axis}[xmin=-6, xmax=6, ymin=0, ymax=2,
	width=1\textwidth,
	height=0.4\textwidth,
	axis x line=middle,
	axis y line=middle, 
	every axis x label/.style={at={(current axis.right of origin)},anchor=west},
	every inner x axis line/.append style={|-latex'},
	every inner y axis line/.append style={|-latex'},
	minor tick num=1,			
	axis equal=true,
	xlabel=$x$, 
	ylabel=$S(x)$,          
	samples=100,
	clip=true,
	]
	\addplot [domain=0:1,blue,line width=2pt] {\x};
	\addplot [domain=1:2,blue,line width=2pt] {0.5};
	\addplot [domain=-2:-1,blue,line width=2pt] {\x+2};
	\addplot [domain=-1:0,blue,line width=2pt] {0.5};
	\addplot [domain=-4:-3,blue,line width=2pt] {\x+4};
	\addplot [domain=-3:-2,blue,line width=2pt] {0.5};
	\addplot [domain=-6:-5,blue,line width=2pt] {\x+6};
	\addplot [domain=-5:-4,blue,line width=2pt] {0.5};
	\addplot [domain=2:3,blue,line width=2pt] {\x-2};
	\addplot [domain=3:4,blue,line width=2pt] {0.5};
	\addplot [domain=4:5,blue,line width=2pt] {\x-4};
	\addplot [domain=5:6,blue,line width=2pt] {0.5};
	\addplot[
	mark=*,
	mark options={fill=black, draw=blue},
	only marks,
	] coordinates {(-6, 0.25) (-5, 0.75) (-4, 0.25) (-3, 0.75) (-2, 0.25) 
	(-1, 0.75) (0, 0.25) (1, 0.75) (2, 0.25) (3, 0.75) (4, 0.25) (5, 0.75) (6, 0.25)};
	\end{axis}
	\end{tikzpicture}
\end{center}

\noindent
\textbf{Ответ:}

\begin{align}
&f(x)=\sum_{n=-\infty}^\infty\frac{\left(\left(-1\right)^n-1\right)\left(2i-\pi n\right)}{4i\pi^2n^2} e^{\tfrac{i2\pi nx}{3}},~ x\ne n; \nonumber\\
&S(2n)=\frac{1}{4}, \nonumber\\
&S(2n+1)=\frac{3}{4},\quad\text{при}~n\in\mathbb{Z}. \nonumber
\end{align}



% ---------------------------- Problem 3----------------------------------
\subsubsection*{\center Задача № 3.}
{\bf Условие.~}\\
Найти резольвенту для интегрального уравнения Вольтерры со следующим ядром 
$$K(x,t)=\frac{2t^2 - t+1}{2x^2-x+1} 2^{\cosh x - \cosh t}.$$

\noindent
{\bf Решение.~}\\
\noindent

Из рекурентных соотношений получаем
$$
\begin{array}{rcl}
K_1(x,t)&=&\displaystyle K(x,t), \\[12pt]
K_2(x,t)&=&\displaystyle\int\limits_t^x K(x,s)K_1(s,t)ds = \int\limits_t^x \frac{2s^2 - s+1}{2x^2-x+1} 2^{\cosh x - \cosh s} \frac{2t^2 - t+1}{2s^2-s+1} 2^{\cosh s - \cosh t} ds =\\ 
&=& K(x,t)\cdot \left( x-t \right),  \\[12pt]
K_3(x,t)&=&\displaystyle\int\limits_t^x K(x,s)K_2(s,t)ds = K(x,t) \int\limits_t^x \left( x-t \right) ds = K(x,t) \frac{\left( x-t \right)^2}{2}.\\[12pt]
K_n(x,t)&=&\displaystyle K(x,t)\frac{\left(x-t\right)^{n-1}}{(n-1)!},\quad n=\mathbb{N}.
\end{array}
$$
Подставляя это выражение для итерированных ядер, найдем резольвенту
$$ 
R(x,t,\lambda)=\sum_{n=0}^\infty \lambda^n K_{n+1}(x,t) =\sum_{n=0}^\infty \lambda^n \frac{\left(x-t\right)^{n}}{n!} = e^{\lambda(x-t)}

$$